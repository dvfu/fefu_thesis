% \iffalse meta-comment
% !TEX program  = xeLaTeX
%<*driver>
\ProvidesFile{fefu_base.dtx}
%</driver>
%<package>\NeedsTeXFormat{LaTeX2e}
%<package>\ProvidesPackage{fefu_base}
%<*package>
    [2019/05/30 LaTeX utility package for FEFU classes]
%</package>
%<*driver>
\documentclass{ltxdoc}
\usepackage[utf8]{inputenc}
\usepackage[T2A]{fontenc}
\usepackage[russian,english]{babel}
\usepackage[a4paper,left=4cm]{geometry}
%\OnlyDescription
\begin{document}
    \DocInput{\jobname.dtx}
\end{document}
%</driver>
%\fi
%
% \title{FEFU utilities}
% \author{}
% \date{}
% \maketitle
% \begin{abstract}
%    The package provides utility macros for FEFU classes.
% \end{abstract}
%
%\CheckSum{0}
%\StopEventually{}
%
% \section{Implementation}
%    \begin{macrocode}
%<*package>
%    \end{macrocode}
% This package requires the following packages:
%    \begin{macrocode}
\RequirePackage{ulem}
\RequirePackage{calc}
\RequirePackage{pbox}
\RequirePackage{array}
\RequirePackage{xifthen}
\RequirePackage{graphicx}
\RequirePackage{fontspec}
\RequirePackage{etoolbox}
\RequirePackage{polyglossia}
%    \end{macrocode}
% The iternal variables are set up here.
%    \begin{macrocode}
\newcommand{\@FEFUschool}{цифровой экономики}
\newcommand{\@FEFUschoolsuffix}{}
\newcommand{\@FEFUgroup}{}
\newcommand{\@FEFUsupervisor}{}
\newcommand{\@FEFUsupervisortitle}{}
\newcommand{\@FEFUconsultant}{}
\newcommand{\@FEFUoffsitesupervisor}{}
\newcommand{\@FEFUoffsitesupervisortitle}{}
\newcommand{\@FEFUreviewer}{}
\newcommand{\@FEFUreviewertitle}{}
\newcommand{\@FEFUdirector}{}
\newcommand{\@FEFUdirectortitle}{}
\newcommand{\@FEFUsecretary}{}
\newcommand{\@FEFUsecretarytitle}{}
\newcommand{\@FEFUdeputy}{}
\newcommand{\@FEFUdeputytitle}{}
\newcommand{\@FEFUdeputyjobtitle}{}
\newcommand{\@FEFUexportsupervisor}{}
\newcommand{\@FEFUexportsupervisortitle}{}
\newcommand{\@FEFUplace}{}
\newcommand{\@FEFUstartdate}{}
\newcommand{\@FEFUenddate}{}
\newcommand{\@FEFUstartyear}{\the\year}
\newcommand{\@FEFUendyear}{\the\year}
\newcommand{\@FEFUpracticetask}{}
\newcommand{\@FEFUfaculty}{}
\newcommand{\@FEFUprogram}{}
\newcommand{\@FEFUpracticetitle}{Учебная ознакомительная практика}
\newcommand{\@FEFUfeedback}{}
\newcommand{\@FEFUschooltitle}{Школа}
\newcommand{\@FEFUschoolreferencetitle}{Школы}
\newcommand{\@FEFUdepartment}{}
\newcommand{\@FEFUdepartmenttitle}{}
\newcommand{\@FEFUdepartmentreferencetitle}{}
\newcommand{\@FEFUadmissiondate}{\@date}
\newcommand{\@FEFUexportcontroldate}{\@date}
\newcommand{\@FEFUsupervisordate}{\@date}
\newcommand{\@FEFUconsultantdate}{\@date}
\newcommand{\@FEFUsecretarydate}{\@date}
%    \end{macrocode}
% \begin{macro}{\setschool}
% \begin{macro}{\setgroup}
% \begin{macro}{\setplace}
% \begin{macro}{\setdepartment}
% \begin{macro}{\setdepartmenttitle}
% \begin{macro}{\setdepartmentreferencetitle}
% \begin{macro}{\setpracticetask}
% \begin{macro}{\setpracticetitle}
% \begin{macro}{\setfaculty}
% \begin{macro}{\setprogram}
% \begin{macro}{\setfeedback}
% \begin{macro}{\setconsultant}
% \begin{macro}{\setschooltitle}
% \begin{macro}{\setschoolreferencetitle}
% \begin{macro}{\setadmissiondate}
% \begin{macro}{\setadmissiondate}
% \begin{macro}{\setadmissiondate}
% \begin{macro}{\setadmissiondate}
% \begin{macro}{\setadmissiondate}
% Sets respective internal variable
%    \begin{macrocode}
\newcommand{\setschool}[1]{\renewcommand{\@FEFUschool}{#1}}
\newcommand{\setschoolsuffix}[1]{\renewcommand{\@FEFUschoolsuffix}{#1}}
\newcommand{\setgroup}[1]{\renewcommand{\@FEFUgroup}{#1}}
\newcommand{\setplace}[1]{\renewcommand{\@FEFUplace}{#1}}
\newcommand{\setdepartment}[1]{\renewcommand{\@FEFUdepartment}{#1}}
\newcommand{\setdepartmenttitle}[1]{\renewcommand{\@FEFUdepartmenttitle}{#1}}
\newcommand{\setdepartmentreferencetitle}[1]{\renewcommand{\@FEFUdepartmentreferencetitle}{#1}}
\newcommand{\setpracticetask}[1]{\renewcommand{\@FEFUpracticetask}{#1}}
\newcommand{\setpracticetitle}[1]{\renewcommand{\@FEFUpracticetitle}{#1}}
\newcommand{\setfaculty}[1]{\renewcommand{\@FEFUfaculty}{#1}}
\newcommand{\setprogram}[1]{\renewcommand{\@FEFUprogram}{#1}}
\newcommand{\setfeedback}[1]{\renewcommand{\@FEFUfeedback}{#1}}
\newcommand{\setconsultant}[1]{\renewcommand{\@FEFUconsultant}{#1}}
\newcommand{\setschooltitle}[1]{\renewcommand{\@FEFUschooltitle}{#1}}
\newcommand{\setschoolreferencetitle}[1]{\renewcommand{\@FEFUschoolreferencetitle}{#1}}
\newcommand{\setadmissiondate}[1]{\renewcommand{\@FEFUadmissiondate}{#1}}
\newcommand{\setexportcontroldate}[1]{\renewcommand{\@FEFUexportcontroldate}{#1}}
\newcommand{\setsupervisordate}[1]{\renewcommand{\@FEFUsupervisordate}{#1}}
\newcommand{\setconsultantdate}[1]{\renewcommand{\@FEFUconsultantdate}{#1}}
\newcommand{\setsecretarydate}[1]{\renewcommand{\@FEFUsecretarydate}{#1}}
%    \end{macrocode}
% \end{macro}
% \end{macro}
% \end{macro}
% \end{macro}
% \end{macro}
% \end{macro}
% \end{macro}
% \end{macro}
% \end{macro}
% \end{macro}
% \end{macro}
% \end{macro}
% \end{macro}
% \end{macro}
% \end{macro}
% \end{macro}
% \end{macro}
% \end{macro}
% \end{macro}
% Iternal switches are declared here.
%    \begin{macrocode}
\newif\if@FEFUfemale
\newif\if@FEFUoffsite
%    \end{macrocode}
% \begin{macro}{\setsupervisor}
% Sets supervisor and supervisor's title iternal variables.
%    \begin{macrocode}
\newcommand{\setsupervisor}[2]{
    \renewcommand{\@FEFUsupervisor}{#1}
    \renewcommand{\@FEFUsupervisortitle}{#2}
}
%    \end{macrocode}
% \end{macro}
% \begin{macro}{\setoffsitesupervisor}
% Sets offsite supervisor and offsite supervisor's title iternal variables.
%    \begin{macrocode}
\newcommand{\setoffsitesupervisor}[2]{
    \renewcommand{\@FEFUoffsitesupervisor}{#1}
    \renewcommand{\@FEFUoffsitesupervisortitle}{#2}
}
%    \end{macrocode}
% \end{macro}
% \begin{macro}{\setreviewer}
% Sets reviewer and reviewer's title iternal variables.
%    \begin{macrocode}
\newcommand{\setreviewer}[2]{
    \renewcommand{\@FEFUreviewer}{#1}
    \renewcommand{\@FEFUreviewertitle}{#2}
}
%    \end{macrocode}
% \end{macro}
% \begin{macro}{\setdirector}
% Sets director and director's title iternal variables.
%    \begin{macrocode}
\newcommand{\setdirector}[2]{
    \renewcommand{\@FEFUdirector}{#1}
    \renewcommand{\@FEFUdirectortitle}{#2}
}
%    \end{macrocode}
% \end{macro}
% \begin{macro}{\setsecretary}
% Sets secretary and secretary's title iternal variables.
%    \begin{macrocode}
\newcommand{\setsecretary}[2]{
    \renewcommand{\@FEFUsecretary}{#1}
    \renewcommand{\@FEFUsecretarytitle}{#2}
}
%    \end{macrocode}
% \end{macro}
% \begin{macro}{\setdeputy}
% Sets deputy and deputy's title iternal variables.
%    \begin{macrocode}
\newcommand{\setdeputy}[3][]{
    \renewcommand{\@FEFUdeputy}{#2}
    \renewcommand{\@FEFUdeputytitle}{#3}
    \renewcommand{\@FEFUdeputyjobtitle}{#1}
}
%    \end{macrocode}
% \end{macro}
% \begin{macro}{\setexportsupervisor}
% Sets export supervisor and their title iternal variables.
%    \begin{macrocode}
\newcommand{\setexportsupervisor}[2]{
    \renewcommand{\@FEFUexportsupervisor}{#1}
    \renewcommand{\@FEFUexportsupervisortitle}{#2}
}
%    \end{macrocode}
% \end{macro}
% \begin{macro}{\setstartdate}
% Sets start date internal variable.
%    \begin{macrocode}
\newcommand{\setstartdate}[2][]{
    \renewcommand{\@FEFUstartdate}{#2}
    \@FEFUifempty{#1}{}{\renewcommand{\@FEFUstartyear}{#1}}
}
%    \end{macrocode}
% \end{macro}
% \begin{macro}{\setenddate}
% Sets end date internal variable.
%    \begin{macrocode}
\newcommand{\setenddate}[2][]{
    \renewcommand{\@FEFUenddate}{#2}
    \@FEFUifempty{#1}{}{\renewcommand{\@FEFUendyear}{#1}}
}
%    \end{macrocode}
% \end{macro}
% \begin{macro}{\setmonthname}
% Sets a name to be displayed for a month
%    \begin{macrocode}
\newcommand{\setmonthname}[2]{\expandafter\def\csname @FEFUmonthname#1\endcsname{#2}}
\setmonthname{01}{января}
\setmonthname{02}{февраля}
\setmonthname{03}{марта}
\setmonthname{04}{апреля}
\setmonthname{05}{мая}
\setmonthname{06}{июня}
\setmonthname{07}{июля}
\setmonthname{08}{августа}
\setmonthname{09}{сентября}
\setmonthname{10}{октября}
\setmonthname{11}{ноября}
\setmonthname{12}{декабря}
%    \end{macrocode}
% \end{macro}
% \begin{macro}{\setfemale}
% \begin{macro}{\setmale}
% Changes endings of some words.
%    \begin{macrocode}
\newcommand{\setfemale}{\@FEFUfemaletrue}
\newcommand{\setmale}{\@FEFUfemalefalse}
%    \end{macrocode}
% \end{macro}
% \end{macro}
% \begin{macro}{\setoffsite}
% \begin{macro}{\setonsite}
% Adds or removes offsite fields.
%    \begin{macrocode}
\newcommand{\setoffsite}{\@FEFUoffsitetrue}
\newcommand{\setonsite}{\@FEFUoffsitefalse}
%    \end{macrocode}
% \end{macro}
% \end{macro}
% \begin{macro}{\@FEFUunderline}
% Used to underline multiline macro.
%    \begin{macrocode}
\useunder{\uline}{\@FEFUunderline}{}
%    \end{macrocode}
% \end{macro}
% \begin{macro}{\@FEFUshortyear}
% Prints short year.
%    \begin{macrocode}
\newcommand*\@FEFUshortyear[1]{\expandafter\@gobbletwo\number\numexpr#1\relax}
%    \end{macrocode}
% \end{macro}
% \begin{macro}{\fefu\_abbreviate}
% Extracts the first argument and puts a dot after it
%    \begin{macrocode}
\ExplSyntaxOn
\cs_new_protected:Npn \fefu_abbreviate:n #1
{
    \exp_args:Nx \str_head:n { #1 }.
}
\ExplSyntaxOff
%    \end{macrocode}
% \end{macro}
% \begin{macro}{\fefu\_abbreviate\_seq}
% Extracts an element from the sequence and abbreviates it
%    \begin{macrocode}
\ExplSyntaxOn
\cs_new_protected:Npn \fefu_abbreviate_seq:Nn #1 #2
{
    \exp_args:Nx \fefu_abbreviate:n { \seq_item:Nn #1 { #2 } }
}
\ExplSyntaxOff
%    \end{macrocode}
% \end{macro}
% \begin{macro}{\@FEFUshortname}
% Prints short name.
%    \begin{macrocode}
\ExplSyntaxOn
\cs_new_protected:Npn \fefu_short_name_helper:n #1
{
    \tl_set:Nx \l_tmpa_tl { #1 }
    \seq_set_split:NnV \l_tmpa_seq { ~ } \l_tmpa_tl
    \int_case:nnF { \seq_count:N \l_tmpa_seq }
    {
        { 2 } { \seq_item:Nn \l_tmpa_seq { 1 } ~ \fefu_abbreviate_seq:Nn \l_tmpa_seq { 2 } }
        { 3 } { \seq_item:Nn \l_tmpa_seq { 1 } ~ \fefu_abbreviate_seq:Nn \l_tmpa_seq { 2 } ~ \fefu_abbreviate_seq:Nn \l_tmpa_seq { 3 } }
    }
    {
        #1
    }
}

\NewDocumentCommand{\@FEFUshortname}{m}{%
    \fefu_short_name_helper:n { #1 }
}
\ExplSyntaxOff
%    \end{macrocode}
% \end{macro}
% \begin{macro}{\@FEFUshortnamereversed}
% Prints short name, surname last
%    \begin{macrocode}
\ExplSyntaxOn
\cs_new_protected:Npn \fefu_short_name_reversed_helper:n #1
{
    \tl_set:Nx \l_tmpa_tl { #1 }
    \seq_set_split:NnV \l_tmpa_seq { ~ } \l_tmpa_tl
    \int_case:nnF { \seq_count:N \l_tmpa_seq }
    {
        { 2 } { \fefu_abbreviate_seq:Nn \l_tmpa_seq { 2 } ~ \seq_item:Nn \l_tmpa_seq { 1 } }
        { 3 } { \fefu_abbreviate_seq:Nn \l_tmpa_seq { 2 } ~ \fefu_abbreviate_seq:Nn \l_tmpa_seq { 3 } ~ \seq_item:Nn \l_tmpa_seq { 1 } }
    }
    {
        #1
    }
}

\NewDocumentCommand{\@FEFUshortnamereversed}{m}{%
    \fefu_short_name_reversed_helper:n { #1 }
}
\ExplSyntaxOff
%    \end{macrocode}
% \end{macro}
% \begin{macro}{\@FEFUacronym}
% Makes an acronym skipping words of length 1
%    \begin{macrocode}
\def\@FEFUacronymhelper#1 #2\relax{%
    \StrLen{#1}[\WordLength]%
    \StrLeft{#1}{1}[\FirstLetter]%
    \ifthenelse{\equal{\WordLength}{1}}{}{\MakeUppercase{\FirstLetter}}%
    \@FEFUifempty{#2}{}{\expandafter\@FEFUacronymhelper#2\relax}%
}
\def\@FEFUacronym#1{\@FEFUifempty{#1}{}{\expandafter\@FEFUacronymhelper\expanded{#1} \relax}}
%    \end{macrocode}
% \end{macro}
% \begin{macro}{\@FEFUgetyear}
% Extracts year from dd.mm.yy or uses default one
%    \begin{macrocode}
\def\@FEFUdropdot#1.\relax{#1}
\newcommand{\@FEFUgetyear}[2][\the\year]{%
    \def\@FEFUgetyearhelper##1.##2.##3\relax{\@FEFUifempty{##3}{#1}{20\@FEFUdropdot##3\relax}}%
    \expandafter\@FEFUgetyearhelper#2.\relax%
}
%    \end{macrocode}
% \end{macro}
% \begin{macro}{\@FEFUgetmonth}
% Gets month from dd.mm or dd.mm.yy
%    \begin{macrocode}
\def\@FEFUgetmonthhelper#1.#2.#3\relax{#2}
\def\@FEFUgetmonth#1{\expandafter\@FEFUgetmonthhelper#1.\relax}
%    \end{macrocode}
% \end{macro}
% \begin{macro}{\@FEFUgetday}
% Gets day from dd.mm or dd.mm.yy
%    \begin{macrocode}
\def\@FEFUgetdayhelper#1.#2\relax{#1}
\def\@FEFUgetday#1{\expandafter\@FEFUgetdayhelper#1\relax}
%    \end{macrocode}
% \end{macro}
% \begin{macro}{\@FEFUifempty}
% Selects \marg{arg2} or \marg{arg3} depending on whether \marg{arg1} is empty or not.
%    \begin{macrocode}
\newcommand{\@FEFUifempty}[3]{%
    \ifthenelse{\equal{#1}{}}{#2}{#3}%
}
%    \end{macrocode}
% \end{macro}
% \begin{macro}{\@FEFUcoalesce}
% Returns first \marg{arg1} if it's not empty otherwise returns \marg{arg2}.
%    \begin{macrocode}
\newcommand{\@FEFUcoalesce}[2]{%
    \@FEFUifempty{#1}{\@FEFUifempty{#2}{}{#2}}{#1}%
}
%    \end{macrocode}
% \end{macro}
% \begin{macro}{\@FEFUgendered}
% Select argument depending on gender setting. \marg{arg1} for female, and \marg{arg2} for male.
%    \begin{macrocode}
\newcommand{\@FEFUgendered}[2]{\if@FEFUfemale#1\else#2\fi}
%    \end{macrocode}
% \end{macro}
% \begin{macro}{\@FEFUsplitandapply}
% Splits on space (default) and applies macro to either split
%    \begin{macrocode}
\newcommand{\@FEFUsplitandapply}[4][ ]{%
    {%
        \def\@FEFUsplitandapplyhelper##1#1##2\relax{#3{##1}#1#4{##2}}%
        \expandafter\@FEFUsplitandapplyhelper#2\relax%
    }%
}
%    \end{macrocode}
% \end{macro}
% \begin{macro}{\@FEFUaddguillemots}
% Adds guillemots around argument
%    \begin{macrocode}
\newcommand{\@FEFUaddguillemots}[1]{\guillemotleft#1\guillemotright}
%    \end{macrocode}
% \end{macro}
% \begin{macro}{\@FEFUremoveprefix}
% Splits on space (default) and remove every but last item
%    \begin{macrocode}
\newcommand{\@FEFUremoveprefix}[2][ ]{
    \def\@FEFUremoveprefixhelper##1#1##2\relax{\@FEFUifempty{##2}{##1}{\@FEFUremoveprefixhelper##2\relax}}%
    \expandafter\@FEFUremoveprefixhelper#2#1\relax
}
%    \end{macrocode}
% \end{macro}
% No title and author by default.
%    \begin{macrocode}
\date{}
\title{}
\author{}
%    \end{macrocode}
% Declare centered column type.
%    \begin{macrocode}
\newcolumntype{C}[1]{>{\centering\arraybackslash}m{#1}}
\newcolumntype{P}[1]{>{\raggedright\arraybackslash}m{#1}}
%    \end{macrocode}
% \begin{macro}{\fefuloadstyle}
% Loads fefu document style
%    \begin{macrocode}
\newcommand{\fefuloadstyle}[2][]{%
    \RequirePackage{#1fefu_style_#2}
}
%    \end{macrocode}
% \end{macro}
% \begin{macro}{\@FEFUcreateLanguageKey}
% Creates a key for the language.
%    \begin{macrocode}
\ExplSyntaxOn

\newcommand{\@FEFUcreateLanguageKey}[1]{
    \keys_define:nn { fefu/languages }
    {
        #1 .tl_set:c = { l_fefu_language_#1_tl }
    }
}

\ExplSyntaxOff
%    \end{macrocode}
% \end{macro}
% Patch polyglossia macro to keep the list of currently enabled languages
%    \begin{macrocode}
\ExplSyntaxOn

\let\@FEFUoriginalsetmainlanguage\setmainlanguage
\let\@FEFUoriginalsetotherlanguage\setotherlanguage
\let\@FEFUoriginalsetotherlanguages\setotherlanguages

\prop_new:N \g_fefu_languages_prop

\cs_new_protected:Npn \fefu_set_main_language:n #1 {
    \prop_put:Nnn \g_fefu_languages_prop { main } {#1}
    \@FEFUcreateLanguageKey{#1}
    \@FEFUoriginalsetmainlanguage{#1}
}

\cs_new_protected:Npn \fefu_set_other_language:n #1 {
    \prop_put:Nnn \g_fefu_languages_prop {other-#1} {#1}
    \@FEFUcreateLanguageKey{#1}
    \@FEFUoriginalsetotherlanguage{#1}
}

\cs_new_protected:Npn \fefu_set_other_languages:n #1 {
    \seq_map_inline:Nn #1 {
        \prop_put:Nnn \g_fefu_languages_prop {other-##1} {##1}
        \@FEFUcreateLanguageKey{##1}
        \@FEFUoriginalsetmainlanguage{##1}
    }
}

\cs_gset_eq:NN \setmainlanguage    \fefu_set_main_language:n
\cs_gset_eq:NN \setotherlanguage   \fefu_set_other_language:n
\cs_gset_eq:NN \setotherlanguages  \fefu_set_other_languages:n
%    \end{macrocode}
% \begin{macro}{\fefuchooselanguage}
% Chooses one of the arguments based on the current main language.
%    \begin{macrocode}
\ExplSyntaxOn

\cs_new_protected:Nn \fefu_choose_language:n
{
    \keys_set:nn { fefu/languages } { #1 }
    \tl_use:c { l_fefu_language_\xpg@main@language _tl }
}

\NewDocumentCommand{\fefuchooselanguage} { m }{\fefu_choose_language:n { #1 }}

\ExplSyntaxOff
%    \end{macrocode}
% \end{macro}
% \begin{macro}{\fefuiflanguage}
% Selects a branch depending on the current main language
%    \begin{macrocode}
\newcommand{\fefuiflanguage}[3]{%
    \ifthenelse{\equal{#1}{\xpg@main@language}}{#2}{#3}%
}
%    \end{macrocode}
% \end{macro}
%    \begin{macrocode}
%</package>
%    \end{macrocode}
% \Finale
